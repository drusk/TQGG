\documentclass[10pt]{article}

\usepackage{amsmath,amssymb}


\begin{document}


\title{Installation Notes}

\maketitle

\section{Background}
TQGG (pronounce like Jacuzzi) is a general purpose interactive program that creates
2-dimensional unstructured grids. The long name is Triangle Quadrilateral Grid Generation and it was
developed from the program Trigrid and its later incarnation GridGen.

For Linux and OSX, the GUI uses Motif and is compiled with gcc and gfortran. 
For Windows, the GUI uses the QUICKWIN routines in Visual Fortran (a combination of
Visual Studio and the Intel fortran compiler). When initially developed 30 odd years 
ago, the program was written in ancient fortran and used a command line interface. With
the advent of more modern GUIs, the graphical interface is written in C and has has 
been separated from the fortran computational code. Hence, this is a mixed-language program.

The source code is in a git repository. The first step is to install git (a revision control program)
on the target system. This is freely available software and may be obtained at \\

\noindent http://git-scm.com/downloads \\

Once git is installed, create a base directory in your user space for the TQGG source; ie, MyPrograms.
Change directory to the directory that has just been created. Then type \\

\noindent git clone https://github.com/rrusk/TQGG.git \\

\noindent which will create a directory ./TQGG and install the code there. In the next sections, 
specific instructions are given for building the program with different operating systems.


\section{Linux}
\subsection{System Requirements}
Linux systems require the Motif development environment, gcc, and gfortran. Normally, gcc is 
available with the code development environment, and gfortran and Motif are available through
the software manager for the particular flavor of Linux being used. The software has been
tested on several versions of openSUSE and Linux Mint. 

\subsection{Building}
On Linux, just type 'make' in the directory TQGG and the executable will be built and put into the
subdirectory 'bin'. Then typing 'bin/TQGG' will execute the program. There are test files in 
the directory 'demodata'.

The program can also be built using the scripts in './TQGGWS/TQGG\_xm'. cd to this directory and
use the scripts to build versions with and without debug support.

\section{Apple OSX}
\subsection{System Requirements}
OSX like Linux requires the Motif development environment, gcc, and gfortran. These are
not normally installed on the proprietary OSX machines. Motif can be installed from several
sources on the web, and gcc and gfortran are available from GNU. You will need to install
Apple's XCode for the compilers. It is interesting getting all this software to dance together,
but the result is worth it. The TQGG code was developed on a Macbook with OSX 10.6,
XCode 4, and Motif 2.1.32-22i.

\subsection{Building}
The program can be built using the scripts in './TQGGWS/TQGG\_mac'. cd to this directory and
use the scripts to build versions with and without debug support. You may have to change the 
path to the Motif code and compilers if they install in a different directory.

\section{MS Windows}
\subsection{System Requirements}
The Windows version requires Visual Studio and a fortran compiler that includes the QUICKWIN
libraries. These libraries are fortran callable subroutines that access a simplified version
of the WIN API. The program was developed using Win XP SP2, Visual Studio 2005, and Intel fortran
compiler 2011.8. Visual Studio 2008 has also been used with 64-bit XP where the project file is
automatically updated to a new format.

\subsection{Building}
The project file TQGG.vfproj can be found in './TQGGWS/TQGG\_vf'. Open this file in Visual Studio
and there will be options to build the 32-bit debug and release versions, and the 64-bit debug 
and release versions.

Unlike the Linux and Mac versions, all the dialog boxes in Windows are modal; ie, the dialog must be 
closed before any other operations can commence. The only solution to this is to program the GUI
directly with the WIN API and not use QUICKWIN.

There have been issues in running these executables on later versions of Windows (7 and possibly 8).
Apparently, there is some security issue with the later versions such that the dialog boxes will
not accept any input so the program hangs. This may not happen with later versions of VS and QUICKWIN that
are built for the later versions of Windows.

\end{document}
